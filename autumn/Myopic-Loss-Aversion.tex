\documentclass[oneside,reqno,letterpaper]{amsart}
\usepackage[plain]{/Users/aden/Library/CloudStorage/Box-Box/latex/adenc}
\usepackage{silence} % for suppressing warnings
\WarningFilter{mdframed}{You have requested package}

\usepackage[natbib=true, style=apa, backend=biber]{biblatex}
\addbibresource{Myopic-Loss-Aversion-ref.bib}


\title[MFE: Behavioral Economics Notes]{MFE: Behavioral Economics Notes}
\author{Aden Chen}


\begin{document}
\maketitle
\tableofcontents

\section{Review}
The essence of the previous few lectures may be summarized as follows:
\begin{enumerate}[label=(\roman*)]
  \item We model an asset as a lottery with multiple uncertain outcomes.
  \item We specified a decision making model: people make decisions according to expected utility (they have a von Neumann-Morgenstern utility function). 
    More specifically, we postulated that they have constant risk aversion and thus have the constant risk aversion utility function
    \[
      u(x) = \frac{x^{1 - \gamma}}{1 - \gamma}.
    \] 
    We estimate the parameter \(\gamma\) using real world decisions.
  \item We construct an economy with identical agents specified above and determine the equilibrium condition: since all agents are identical, at equilibrium there is no trade.
    The price of the asset is then the price at which there is no trade.
  \item We compare the real-world asset price to that of the model and found the former to be much, much higher --- the asset premium puzzle.
    Alternatively, we may think of the puzzle as follows:
    to obtain an asset price comparable to that observed in the real world, we need an absurdly large \(\gamma\).
\end{enumerate}


\section{Understanding the parameter gamma}
Consider the lottery \([11, 10; .5, .5]\) (that is, the individual gets \(11\) dollars with probability \(0.5\) and \(10\) dollars with probability \(0.5\)) in the standard expected utility framework.
The following table gives the minimum \(\gamma\) under which the agent rejects the lottery at different levels of wealth \(w\).

\begin{tabular}{ c c } 
  Rejects at \(w=\) & \(\gamma > \) \\ 
\hline
  \num{1000} & 9 \\ 
  \num{2000} & 18 \\ 
  \num{5000} & 45 \\ 
\end{tabular}

Estimate your own \(\gamma\) by considering the lottery \([11, 10; .5, .5]\) and your own wealth!
Given that many would reject such a gamble, we would expect \(\gamma\) to be relatively large.


Next, consider the lottery \([50000, 100000; .5, .5]\) (the individual gets \(50000\) dollars with probability \(0.5\) and \(100000\) dollars with probability \(0.5\)).

\begin{tabular}{ c c } 
  \(\gamma\) & \(W_{CE}\) \\ 
\hline
  1 & \num{70711} \\ 
  2 & \num{66667} \\ 
  5 & \num{58566} \\ 
  10 & \num{53991} \\ 
  30 & \num{51209} \\ 
\end{tabular}\footnote{Table taken from \textcite{Schilbach2020Lecture}. }
Estimate again your \(\gamma\) and compare it to the previous estimate.
Note the disparity!

In fact, we can show mathematically that if a risk averse expected utility maximizer rejects the small gamble \([11, 10; .5, .5]\) at any \(w\), then she will reject a \([-100, \infty]\) lottery.\footnote{
\[
  10^{-4} > (10 / 11)^{100} \implies u'(w + 2100) \leq 10^{-4} u'(w) . 
\] 
}

Now, one immediate idea to fix our decision model would be to say that \(\gamma\) (or even the utility function \(U\)) is not fixed.
Preferences and perceptions of / attitudes to risk changes.
This is, however, a dangerous approach.
We are pushing sweep something under the rug, so to speak.
Unless we develop we develop a theory of how the utility function or \(\gamma\) changes, we can, by changing \(U\) or \(\gamma\), explain everything --- and thus nothing.

Before moving on to a potential resolution, let us first get a fuller picture of why expected utility theory might fail.
We've already seen the problem of risk aversion above --- people seem to be exhibit radically different risk preferences in face of different gambles.
There are but only more problems:


\section{Problems with Expected Utility Theory}
\subsection{Loss Aversion}
Compare the following lotteries:
\begin{itemize}
  \item \([4000; .8]\) and \([3000, 1]\) (that is, getting \(4000\) dollars with probability \(0.8\) or a sure gain of \(3000\) dollars) --- most prefer the latter
  \item \([-4000; .8]\) and \([-3000, 1]\) --- most prefer the former
\end{itemize}
People seem to be risk-averse for gains and risk-loving for losses.
\footnote{\Textcite{Schilbach2020Lecture}. }


\subsection{Framing and Reference Point}
Compare the following:
\begin{itemize}
  \item \([240, 1]\) and \([1000; .25]\) (a sure gain of \(240\) dollars or a gain of \(1000\) dollars with probability \(0.25\))
  \item \([-750, 1]\) and \([-1000; .75]\). 
\end{itemize}
Most preferred the former in the first case and the latter in the second case, the this combination (\([-760, 240; .75, .25]\)) is in fact dominated by choosing the former in the first case and the former in the second case (\([-750, 250; .75, .25]\))!

How the lotteries are framed matters.
(Some call the aggregation of multiple lotteries ``bracketing''.)


\subsection{Endowment Effect, Status Quo Bias}
We can measure how valuable an object is to someone by asking:
\begin{itemize}
  \item how much they are willing to pay (WTP) to get the object; or
  \item how much they are willing to accept (WTA) as compensation for giving up the object.
\end{itemize}
(You might also know them as \(EV\) and \(CV\) if you've taken Econ here.)

In the expected utility framework, we would expect \(WTP \approx WTA\).

A \Textcite{Kahneman1991Endowment} experiment, however, showed otherwise:
\begin{itemize}
  \item Cornell mugs given randomly to half of the students. 
  \item On average we expect half of the mugs to be traded.
  \item Typically \(WTP \approx 1 / 2 WTA\) and around half of trading volume was observed. 
\end{itemize}
Owning (being endowed) the mug seems to cause the students to assign it a higher value.
This is termed the endowment effect.



\section{Prospect Theory}
Prospect theory is the leading alternative to expected utility theory.

Properties:
\begin{itemize}
  \item A reference point: \(U\) is a function not of wealth, but of the change in wealth.
  \item Loss aversion. 
\end{itemize}
\[
  V(G) = \sum \pi(p) v(x),
\] 
where
\begin{itemize}
  \item \(\pi(p)\) is the probability weighting function,
  \item \(v(x)\) is the value function.
\end{itemize}
The interested should Google.

In \textcite{Benartzi1995Myopic}, 
\[
  v(x) = \begin{cases}
    x^{\alpha}, & x \geq 0, \\ 
    -\lambda (-x)^{\beta}, & x \geq 0. 
  \end{cases}
\] 
with the previously\footnote{\textcite{Tversky1992Advances}} estimated parameters \(\alpha = \beta = 0.88\) and \(\lambda = 2.25\). 


\begin{itemize}
  \item \(\lambda\) captures loss aversion.
  \item concave when \(x \geq 0\) and convex when \(x \leq 0\) (Risk aversion)
  \item \(x\) is change not absolute value. (reference dependent; risk aversion)
  \item \(\alpha = \beta < 1\). Diminishing sensitivity. 
  \item There is a sense of framing: gains and losses change when the relevant time period changes.
\end{itemize}

% Recall in expected utility theory, the utility of a gamble \(G\) is given by
% \[
%   V(G) = \E[x] = \sum \Pr[x_i] v(x_i) . 
% \] 


\subsection{Myopic Loss Aversion}
The pairing of prospect theory with framing (frequent evaluation of portfolio) is called myopic loss aversion. 

Previous research: how risk averse would the representative investor have to be to explain the historical equity premium. 
\Textcite{Benartzi1995Myopic}: given these estimated parameters of prospect theory, what evaluation period is consistent with the equity premium? 

One year. This is highly plausible, given the existence of annual reports, and the tex filings, when one would gain a comprehensive evaluation. 


\subsection{Which aspects of prospect theory drive the results?}
Loss aversion. 
Using the identity function as the probability weighting function, 11--12 months to 10 months. 
Using piecewise linear utility function with loss aversion factor \(2.25\), 8 months.


\subsection{Objections}
A potential objection. 
Individual decision making vs.\ organization decision making. 
The issue of agency. 




\section{Additional Topics (if time permits)}
\begin{itemize}
  \item Hyperbolic Discounting. 
  \item Mental Accounting. 
\end{itemize}





\printbibliography





\end{document}




